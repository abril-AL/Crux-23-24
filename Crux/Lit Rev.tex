\documentclass[a4paper]{article}

\usepackage[english]{babel}
\usepackage[utf8x]{inputenc}
\usepackage{amsmath}
\usepackage{graphicx}
\usepackage[colorinlistoftodos]{todonotes}
\usepackage[margin=1.2in]{geometry}

\usepackage{blindtext}
\usepackage[super]{nth}
\usepackage{ulem}
\usepackage{xcolor, soul}
\sethlcolor{yellow}
\usepackage{pgfplots}
\pgfplotsset{compat=1.5}
\usepackage{amssymb}
\usepackage{amsmath,amssymb}
\usepackage{enumitem}


\newcommand*{\Rn}{$\mathbb{R}^n$}
\newcommand*{\x}{$\Vec{x}$}
\newcommand*{\y}{$\Vec{y}$}
\newcommand*{\vc}{$\Vec{vc}$}



\title{Crux 2023-20 24}
\author{Literature Review}
\date{02/17/24}

\begin{document}
\maketitle


\subsection*{Introduction}

This Literature Review is focused on identifying what EEG channels to use for our project. The focus in on measuring concentration through measuring beta power and averages.

I also looked a bit further into the ways to avoid electrical noise with air tube headphones. In addition i looked into the effect of binaural beats on EEG signals.

\subsection*{Source 1 - Analysis of EEG Signals for the Estimation of Concentration Level of Humans}

\begin{itemize}
    \item Introduction
          \begin{itemize}
              \item Characteristics extracted from raw EEG using FFT, mean, standard deviation, median and mean root square. Concentration level determined by comparing extracted features.
              \item Prefrontal lobe responsible for attention and concentration, judgement and motor skills.
              \item Dominant high frequencies indicate ppl are alert.
              \item Intensities(Hz) split into bands
              \item Beta Waves - 14-30 Hz, awake, alert and perceptive state
          \end{itemize}
    \item Methods
          \begin{itemize}
              \item EEG 10-20 amplification method
              \item 3 electrodes (not given??)
              \item notch filter
              \item participants aged 20-23
              \item task of reading a book, puzzle, counting in reverse for 10 minutes
          \end{itemize}
    \item Feature Extraction
          \begin{itemize}
              \item FFT, average, SD, determination of variance
              \item raw signal $=>$ frequency domain using FFT
              \item signal classified among the 5 brain waves
              \item Max mean, RMS, SD, and Var observed in first 2 mnutes, and min in last 2 minutes
          \end{itemize}
    \item Results and Discussion
          \begin{itemize}
              \item FFT is effective for feature extraction
          \end{itemize}
\end{itemize}

Overall this paper was okay, i dont think its a very reliable source, but it at least offered some simple methods for feature extraction.

\subsection*{Source 2 - Measurement of Concentration Duration on Reading Activity: EEG Analysis with OpenBCI Ganglion Board}

\begin{itemize}
    \item Abstract
          \begin{itemize}
              \item Someone who is concentrating will be at a frequency of 15-18Hz
              \item Reading requires concentration
              \item concentration power while reading is indicated through the duration of concentration
              \item 15 minute recordings usign OpenBCI
              \item sampling rate of 200Hz, max impedance of 15 ohms
              \item Processed with EEGLab and EDGF Browser
              \item 3 classes for results: low, non, and high concentration
          \end{itemize}
    \item Introduction
          \begin{itemize}
              \item concentration is at a frequency of 15-18Hz which is in the beta range
              \item concentration power can be described based on the duration of concentration
          \end{itemize}
    \item Methods
          \begin{itemize}
              \item two main theories are used: neurolinguistic theory and brain wave theory
          \end{itemize}
    \item Findings
          \begin{itemize}
              \item EEG with the brain wave range at 13-40Hz (Beta Rhythm) is characterized as the optimal concentration condition, especially at 15-18 Hz.
              \item If it is above 22 Hz even up to 40 Hz (Gamma Rhythms), EEG data is characterized as high concentration (focus), but can have anxiety / anxiety impact.
              \item Meanwhile, if under 13 Hz it shows diminished concentration and towards a relaxed state and toward a reduction of awareness conditions
              \item average concentration in 15 was 11 minutes 3 seconds: optimal con. 3m19s, low con. 6m9s
          \end{itemize}
\end{itemize}

This paper is much clearer, still not a great academic source, but at least offers more on how to classify concentration.

\subsection*{Source 3 - Influence of Binaural Beats on EEG Signal}

\begin{itemize}
    \item Abstract
          \begin{itemize}
              \item requency of
                    f = 10 Hz. The left ear was exposed to a signal with a frequency of 100 Hz, and the right ear — to a signal with a frequency of 110 Hz, the acoustic pressure level SPL = 73 dB
              \item ecrease of average amplitudes of spectral density function of EEG strength signal for alpha and beta frequency ranges
              \item amplitude of spectral density function of the strength has increased in theta frequency range
          \end{itemize}
    \item Introduction
          \begin{itemize}
              \item Binaural beats are a result of superposition of neuron discharge coming from the left and right ear on a suitable level of the hearing route
              \item Two signals are connected in the brain, the result being
                    a sensation of hearing a third signal — with a frequency of a signal provided to the left and the right earcalled binaural beats
              \item reticular system decides about lucidity, concentration and consciousness, changes the brain wave activity so that it is adjusted to the stimulation of the beat signals
              \item maintain homeostasis
              \item subjects listened through stereo headphones to pure tones designed to produce delta and theta binaural beats
              \item analysis of the EEG data involved computing the changes in the percentages of total EEG amplitudes, comparing the conditions of waking rest, binaural-beat stimulus periods,and a second period of rest
              \item ubjects generated significantly less alpha and beta, and more delta and theta freq. brainwaves
              \item binaural beats may be associated with
                    reduced EEG arousal
              \item other research observed reductions in the percentages of occipital alpha (bipolar O1-O2) were significant
              \item reductions in the percentages of central delta(bipolar C3-C4) were similarly significant
          \end{itemize}
    \item Methods
          \begin{itemize}
              \item binaural beats with an acoustic pressure level of SPL = 73 dB,
              \item frequencies: the right ear — 110 Hz,left ear - 100 Hz
              \item total duration of the experiment was 35 min, 20 with stimulus
              \item played on stereo headphones
              \item EEG cap was fitted in accordance with a standard 10/20 system
              \item electrodes are placed along sagittal line of the head \textbf{(5 on the left side: Fp1, F3, C3, P3, O1 and 5 on the right side: Fp2, F4, C4, P4, O2 and a reference electrode on the OP, Pz)}
              \item The initial 5 min was without the binaural beats exposition, 20 min with the signal exposition and 10 min without the exposition.
          \end{itemize}
    \item Results
          \begin{itemize}
              \item analysis of a spectral density function of EEG strength signal
              \item occurred a component in EEG signal morphology, with a frequency of the presented binaural beats
              \item assumed an EEG signal frequency division used in the electro encephalography: beta — from 12.0 to 29.9 Hz, alpha — from 8.0 to 11.9 Hz, theta — from 4.0 to 7.9 Hz, delta — from 0.5 to 3.9 Hz
          \end{itemize}
    \item Conclusion
          \begin{itemize}
              \item binaural beats appear to engender changes in cortical arousal
              \item follow-up effect observed in 4 participants
              \item statistically significant fall of average amplitudes of a spectral density function of EEG strength signal for alpha (p < 0.001) and beta (p < 0.001) frequency ranges. For theta (p < 0.0231) frequency range, however, there was noted an increase of a spectral density function of EEG strength signal.
              \item significant reduction of alpha rhythm (8-12 Hz) while increasing narrow band share in the range 9.9-10.1 Hz
              \item
          \end{itemize}
\end{itemize}

This paper was much better, methods were much clearer and thought out. This study was focused on the effect overall on EEG rather than on just concentration, however, they did go over the effect on beta waves, which is of interest. Since binaural beats seem to decrease beta power, we may see less strong indicators of concentration in out study.

\subsection*{Source 4 - Auditory beat stimulation and its effects on cognition and mood states}

Recent study seeking to investigate these effects in the alpha and beta ranges has reported no significant changes

participants exhibited reduced alpha activity during the binaural beat on-phase compared to the off-phase

data suggest that application of binaural beats in theta, alpha, delta, and beta frequencies is able to alter functional connectivity between brain regions

\subsection*{Source 5 - Alternative headphones for patient noiseprotection and communication in PET-MR studies of the brain}

Introduction

\begin{itemize}
    \item Due to the high level of noise generated by the gradient coils in the strong magnetic field during a magnetic resonance imaging (MRI) examination, noise protection for the patient is mandatory
    \item Also need communication from operator to patient
    \item usually achieved with nonmagnetic headphones (pneumatic headphones \$\$), but need the space, also a constraint of MR compatability
    \item customised earphones with very low gamma ray attenuation
\end{itemize}

\noindent
Results / Conclusiom

\begin{itemize}
    \item In contrast to the use of headphones, qualitative and quantitative errors in the PET images were avoided
    \item local biased image quantification induced by standard headphones in PET images, acquired during simultaneous PET-MR imaging is avoided
    \item no influence on MR image quality
\end{itemize}

This study supports the need for non-electrical headphones when studying electrical signals in the brain. While images may not benefit, PET images did experience less noise.

\subsection*{Source 6 - Analysis of EEG activity in response to binaural beats with different frequencies}

\noindent
Abstract / Introduction
\begin{itemize}
    \item employed Relative Power (RP), Phase Locking Value (PLV) and Cross-Mutual Information (CMI) to track EEG changes during BB stimulations.
    \item Five-minute BBs with four different frequencies were tested: delta band (1 Hz), theta band (5 Hz), alpha band (10 Hz) and beta band (20 Hz)
    \item RP increase in theta and alpha bands and decrease in beta band during delta and alpha BB stimulations
    \item RP decreased in beta band during theta BB, while RP decreased in theta band during beta BB
    \item  no clear brainwave entrainment effect was identified
    \item suggests that the mechanism of BB-brain interaction is worth further study
\end{itemize}

Introduction
\begin{itemize}
    \item When a sound with a steady intensity and frequency was presented to one ear and another with the same intensity but slightly different frequency, the brain producea pulsations in the amplitude and localization that is the same with the perceived sounds
    \item BB fundamental frequency: $$\dfrac{\text{Freq}_{left} + \text{Freq}_{right}}{2}$$
    \item Modulation frequency: $$|Freq_{left} - freq_{right}|$$
    \item difference must be small ($\leq30Hz$)
    \item suggested that tones with a frequency from 200 to 900 Hz were more effective
    \item  basic theoretical assumption was that the human brain had a tendency to change its dominant EEG frequency towards the frequency of external stimulus by entraining the brain to synchronize neural activity with BB stimuli or other external stimulations (not supported by other studies)
    \item may not change if duration is too short or averaging wiped out changes
\end{itemize}

Methods
\begin{itemize}
    \item participants had no caffeine / alch / drugs
    \item told to be quiet and concentrate
    \item to avoid influence of any external noise, a sound isolating earphone was used
    \item \textbf{brain regions were parcellated into six areas: frontal (Fp1, Fp2, F3, Fz and F4), left temporal (F7, T3 and T5), right temporal (F8, T4 and T6), central (C3, Cz and C4), parietal (P3, Pz and P4) and occipital (O1 and O2) areas.}
    \item EEG data were \textbf{preprocessed} by removing the ocular artifact and possible interference from head and muscle movements as well as 50 Hz frequency interference
    \item \includegraphics*[scale=.35]{Fig1.png}
    \item relative power of dominant EEG frequency of 13 subjects showed no enhancement
    \item by tracking the changes of relative power (RP) instead of power, discovered several interesting phenomena. In delta and alpha BBs, RP of theta and alpha bands decreased, while RP of beta band increased.
    \item enhanced beta power in EEG may lead to improved mental focuses
    \item \textbf{delta and alpha BBs can lead to decreased theta and alpha activity and increased beta activity. This may suggest an approach to reduce drowsiness and improve mental focus, which in turn can treat attention-deficit hyperactivity disorder}
    \item \includegraphics*[scale=.35]{Fig4.png}
\end{itemize}

This study implies that BB and their affect on the brain and behavior are worth further study, and which BB increase beta activity.


\subsection*{Source 7 - Determination of the Concentrated State Using Multiple EEG Channels}

Abstract
\begin{itemize}
    \item concentration index was compared during resting and concentrating periods
    \item locations the frontal lobe (Fp1 and Fp2) showed a clear increase of the concentration index during concentration
    \item Fp1, Fp2 , T3, T4 , C3 , C4, O1 and O2 following the International 10-20 system
    \item 30s rest, 60s concentration watching red dot on monitor
    \item Index of concentration: Index = powr of $[{\beta + SMR} / \theta]$
    \item counted occurences of times index was higher than 30\% above average index during rest
    \item most occurences 11~16 after concentration period started
\end{itemize}

Results
\begin{itemize}
    \item Fp1 and Fp2 in frontal lobe showed significant increase in occurences of concentration for both participants
\end{itemize}

\subsection*{Source 8 - }
\end{document}